\documentclass[12pt]{article}
%NOTE: This report format is 

\newcommand{\reporttitle}{Computer Project: Problem 5}
\newcommand{\reportauthorOne}{Tom BEAUVE}
\newcommand{\cidOne}{your id number}
\newcommand{\reportauthorTwo}{Benjamin BOCK}
\newcommand{\cidTwo}{S2304467}
\newcommand{\reportauthorThree}{Robby IRAKIZA}
\newcommand{\cidThree}{your id number}
\newcommand{\reportauthorFour}{Maxime LAYALLE}
\newcommand{\cidFour}{your id number}
\newcommand{\reportauthorFive}{Lilian STEIMETZ}
\newcommand{\cidFive}{your id number}
\newcommand{\reportauthorSix}{Hugo VANGEEBERGEN}
\newcommand{\cidSix}{your id number}
\newcommand{\reporttype}{Coursework}
\bibliographystyle{plain}

% include files that load packages and define macros
\usepackage{fontspec}
\usepackage{newtxtext,newtxmath} % Times New Roman pour texte et maths



% Packages utiles (ajoutez ceux dont vous avez besoin)
\usepackage[letterpaper,hmargin=2.8cm,vmargin=2.0cm,includeheadfoot]{geometry}
\usepackage{textpos}
\usepackage{natbib}
\usepackage{stackengine}
\usepackage{tabularx,longtable,multirow,subfigure,caption}%hangcaption
\usepackage{fncylab} %formatting of labels
\usepackage{fancyhdr}
\usepackage{color}
\usepackage[tight,ugly]{units}
\usepackage{url}
\usepackage{float}
\usepackage[english]{babel}
\usepackage{amsmath}
\usepackage{graphicx}
\usepackage[colorinlistoftodos]{todonotes}
\usepackage{dsfont}
\usepackage{epstopdf} % automatically replace .eps with .pdf in graphics
\usepackage{backref}
\usepackage{array}
\usepackage{etoolbox}
\usepackage{enumerate} % for numbering with [a)] format 
\usepackage{tcolorbox}
\usepackage{graphicx} % Pour insérer des images
\usepackage{tocloft}  % Pour personnaliser la liste des figures

% table of content
\renewcommand{\tableofcontentsname}{Table of Contents}

% table of figures
\renewcommand{\listfigurename}{Table of Figures}

% various theorems
\usepackage{ntheorem}
\theoremstyle{break}
\newtheorem{lemma}{Lemma}
\newtheorem{theorem}{Theorem}
\newtheorem{remark}{Remark}
\newtheorem{definition}{Definition}
\newtheorem{proof}{Proof}

% example-environment
\newenvironment{example}[1][]
{ 
\vspace{4mm}
\noindent\makebox[\linewidth]{\rule{\hsize}{1.5pt}}
\textbf{Example #1}\\
}
{ 
\noindent\newline\makebox[\linewidth]{\rule{\hsize}{1.0pt}}
}

\setlength{\parindent}{0em}  % indentation of paragraph

\setlength{\headheight}{14.5pt}
\pagestyle{fancy}
\fancyfoot[ER,OR]{\thepage}%Page no. in the left on
                                %odd pages and on right on even pages
\fancyfoot[OC,EC]{\sffamily }
\renewcommand{\headrulewidth}{0.1pt}
\renewcommand{\footrulewidth}{0.1pt}
\captionsetup{margin=10pt,font=small,labelfont=bf}

%--- chapter heading
\def\@makechapterhead#1{%
  \vspace*{10\p@}%
  {\parindent \z@ \raggedright
    \interlinepenalty\@M
    \Huge \bfseries 
    \thechapter \space\space #1\par\nobreak
    \vskip 30\p@
  }}

%---chapter heading for \chapter*  
\def\@makeschapterhead#1{%
  \vspace*{10\p@}%
  {\parindent \z@ \raggedright
    \interlinepenalty\@M
    \Huge \bfseries  
    #1\par\nobreak
    \vskip 30\p@
  }}

% %%%%%%%%%%%%% boxit
\def\Beginboxit
   {\par
    \vbox\bgroup
	   \hrule
	   \hbox\bgroup
		  \vrule \kern1.2pt %
		  \vbox\bgroup\kern1.2pt
   }

\def\Endboxit{%
			      \kern1.2pt
		       \egroup
		  \kern1.2pt\vrule
		\egroup
	   \hrule
	 \egroup
   }	

\newenvironment{boxit}{\Beginboxit}{\Endboxit}
\newenvironment{boxit*}{\Beginboxit\hbox to\hsize{}}{\Endboxit}

\allowdisplaybreaks

\makeatletter
\newcounter{elimination@steps}
\newcolumntype{R}[1]{>{\raggedleft\arraybackslash$}p{#1}<{$}}
\def\elimination@num@rights{}
\def\elimination@num@variables{}
\def\elimination@col@width{}
\newenvironment{elimination}[4][0]
{
    \setcounter{elimination@steps}{0}
    \def\elimination@num@rights{#1}
    \def\elimination@num@variables{#2}
    \def\elimination@col@width{#3}
    \renewcommand{\arraystretch}{#4}
    \start@align\@ne\st@rredtrue\m@ne
}
{
    \endalign
    \ignorespacesafterend
}
\newcommand{\eliminationstep}[2]
{
    \ifnum\value{elimination@steps}>0\leadsto\quad\fi
    \left[
        \ifnum\elimination@num@rights>0
            \begin{array}
            {@{}*{\elimination@num@variables}{R{\elimination@col@width}}
            |@{}*{\elimination@num@rights}{R{\elimination@col@width}}}
        \else
            \begin{array}
            {@{}*{\elimination@num@variables}{R{\elimination@col@width}}}
        \fi
            #1
        \end{array}
    \right]
    & 
    \begin{array}{l}
        #2
    \end{array}
    &%                                    moved second & here
    \addtocounter{elimination@steps}{1}
}
\makeatother

%% Fast macro for column vectors
\makeatletter  
\def\colvec#1{\expandafter\colvec@i#1,,,,,,,,,\@nil}
\def\colvec@i#1,#2,#3,#4,#5,#6,#7,#8,#9\@nil{% 
  \ifx$#2$ \begin{bmatrix}#1\end{bmatrix} \else
    \ifx$#3$ \begin{bmatrix}#1\\#2\end{bmatrix} \else
      \ifx$#4$ \begin{bmatrix}#1\\#2\\#3\end{bmatrix}\else
        \ifx$#5$ \begin{bmatrix}#1\\#2\\#3\\#4\end{bmatrix}\else
          \ifx$#6$ \begin{bmatrix}#1\\#2\\#3\\#4\\#5\end{bmatrix}\else
            \ifx$#7$ \begin{bmatrix}#1\\#2\\#3\\#4\\#5\\#6\end{bmatrix}\else
              \ifx$#8$ \begin{bmatrix}#1\\#2\\#3\\#4\\#5\\#6\\#7\end{bmatrix}\else
                 \PackageError{Column Vector}{The vector you tried to write is too big, use bmatrix instead}{Try using the bmatrix environment}
              \fi
            \fi
          \fi
        \fi
      \fi
    \fi
  \fi 
}  
\makeatother

\robustify{\colvec} % various packages needed for maths etc.
\input{notation} % short-hand notation and macros


%%%%%%%%%%%%%%%%%%%%%%%%%%%%

\begin{document}
% front page
% Last modification: 2016-09-29 (Marc Deisenroth)
% Modification for UW: 2017-05-22 (jphickey)
\begin{titlepage}

\newcommand{\HRule}{\rule{\linewidth}{0.5mm}} % Defines a new command for the horizontal lines, change thickness here


%----------------------------------------------------------------------------------------
%	LOGO SECTION
%----------------------------------------------------------------------------------------



\begin{center} % Center remainder of the page

%----------------------------------------------------------------------------------------
%	HEADING SECTIONS
%----------------------------------------------------------------------------------------

\includegraphics[width = 15cm]{./figures/uliege_faculte_sciencesappliquees_logo_rvb}\\[1.5cm] 
\textbf{\textsc{\Large MECA0036-2 Finite Element Method}}\\[1.0cm] 
\textsc{\Large University of Liège}\\[0.5cm] 
\textsc{\large Faculty of Applied Sciences}\\[0.95cm] 

%----------------------------------------------------------------------------------------
%	TITLE SECTION
%----------------------------------------------------------------------------------------

\HRule \\[0.4cm]
{ \huge \bfseries \reporttitle}\\ % Title of your document
\HRule \\[1.5cm]
\end{center}
%----------------------------------------------------------------------------------------
%	AUTHOR SECTION
%----------------------------------------------------------------------------------------

%\begin{minipage}{0.4\hsize}

\begin{center}
    \begin{minipage}{0.45\linewidth} % Bloc à gauche
        \raggedright % Aligné à gauche
        \normalsize \textit{Professor:} \\
            \small Jean-Philippe PONTHOT
    \end{minipage}
    \begin{minipage}{0.45\linewidth} % Bloc à droite
        \raggedleft % Aligné à droite
        \normalsize \textit{Authors:} \\
        \begin{small}
            \reportauthorOne~(\cidOne)\\
            \reportauthorTwo~(\cidTwo)\\
            \reportauthorThree~(\cidThree)\\
            \reportauthorFour~(\cidFour)\\
            \reportauthorFive~(\cidFive)\\
            \reportauthorSix~(\cidSix)\\
        \end{small}
    \end{minipage}
\end{center}


\vspace{4cm}
\makeatletter
Date: \@date 

\vfill % Fill the rest of the page with whitespace



\makeatother

\end{titlepage}




%%%%%%%%%%%%%%%%%%%%%%%%%%% table of content
%If a table of content is needed, simply uncomment the following lines
\renewcommand{\contentsname}{Table of Content}

\tableofcontents
\newpage
\listoffigures
\newpage
\listoftables
\newpage

%%%%%%%%%%%%%%%%%%%%%%%%%%%% Main document
\section{Introduction}



\section{Strength of Materials}



\begin{equation}
\underline{\mathbf{u}} =
\begin{pmatrix}
    u(x,y,z) \\
    v(x,y,z) \\
    w(x,y,z)
\end{pmatrix}, \quad
\underline{\underline{\boldsymbol{\sigma}}} =
\begin{pmatrix}
    \sigma_{xx} & \sigma_{xy} & \sigma_{xz} \\
    \sigma_{yx} & \sigma_{yy} & \sigma_{yz} \\
    \sigma_{zx} & \sigma_{zy} & \sigma_{zz}
\end{pmatrix}, \quad
\underline{\underline{\boldsymbol{\varepsilon}}} =
\begin{pmatrix}
    \varepsilon_{xx} & \varepsilon_{xy} & \varepsilon_{xz} \\
    \varepsilon_{yx} & \varepsilon_{yy} & \varepsilon_{yz} \\
    \varepsilon_{zx} & \varepsilon_{zy} & \varepsilon_{zz}
\end{pmatrix}
\end{equation}

\subsection{Hypotheses for the general equations} \label{sec: SoM general hyp}
\begin{itemize}
    \item Linear Elastic material
    \item Isotropic material
    \item Continuous Medium
\end{itemize}

\subsection{General Equations}

\subsubsection{Equilibrium Equations}
\paragraph{Volume Equilibrium}
\begin{equation}
    \frac{\partial\sigma_{ij}}{\partial x_j} + F_i = 0
    \label{eqn: SoM volume equilibrium}
\end{equation}
\paragraph{Surface equilibrium}
\begin{equation}
    T_j = n_i \sigma_{ij}
    \label{eqn: SoM surface equilibrium}
\end{equation}
\paragraph{Moment equilibrium}
\begin{equation}
    \sigma_{ij} = \sigma_{ji}
\end{equation}

\subsubsection{Displacement-Strain Relation}
Compatibility Equation
\begin{equation}
    \underline{\underline{\boldsymbol{\varepsilon}}} = \frac{1}{2} ( \nabla \underline{u} + (\nabla \underline{u})^T )  
    \quad ; \quad  
    \varepsilon_{ij} = \frac{1}{2} \left( \frac{\partial u_i}{\partial x_j} + \frac{\partial u_j}{\partial x_i} \right)
    \label{eqn: SoM displacement-stress }
\end{equation}
We note that the strain tensor is symmetric, as $\varepsilon_{ij} = \varepsilon_{ji}$\\

Saint venant ( optional ? ), that the strain field has to respect
\begin{equation}
    D_{kk} \, \varepsilon_{ij} + D_{ij} \, \varepsilon_{kk} - D_{jk} \, \varepsilon_{ik} - D_{ik} \, \varepsilon_{jk} = 0
\end{equation}

\noindent where 
\begin{equation}
    D_{ij} = \frac{\partial^2}{\partial x_i \partial x_j}
\end{equation}


\subsubsection{Stress-Strain Relation}
Hooke's law : Elastic Linear 
\begin{equation}
    \underline{\underline{\boldsymbol{\sigma}}} = \underline{\underline{\underline{\underline{\boldsymbol{H}}}}} \space \underline{\underline{\boldsymbol{\varepsilon}}}
    \quad ; \quad 
    \sigma_{ij} = H_{ijkl} \space \varepsilon_{kl}
    \label{eqn: SoM stress-strain}
\end{equation}
Using : symmetric stress and strain tensors and isotropic material assumption, we can get 
\begin{equation}
    \begin{pmatrix}
    \varepsilon_{11} \\
    \varepsilon_{22} \\
    \varepsilon_{33} \\
    \gamma_{12} \\
    \gamma_{23} \\
    \gamma_{13}
    \end{pmatrix}= \frac{1}{E}\begin{pmatrix}
    1 & - \nu & -\nu & 0 & 0 & 0 \\
    -\nu & 1 & -\nu & 0 & 0 & 0 \\
    -\nu & -\nu & 1 & 0 & 0 & 0 \\
     &  &  & 2(1+\nu) & 0 & 0 \\
     &  \text{SYM}&  &  & 2(1+\nu) & 0 \\
     &  &  &  &  & 2(1+\nu)
    \end{pmatrix}
    \begin{pmatrix}
    \varepsilon_{11} \\
    \varepsilon_{22} \\
    \varepsilon_{33} \\
    \sigma_{12} \\
    \sigma_{23} \\
    \sigma_{13}
    \end{pmatrix}
\end{equation}

\begin{equation}
    \begin{pmatrix}
    \sigma_{11} \\
    \sigma_{22} \\
    \sigma_{33} \\
    \sigma_{12} \\
    \sigma_{23} \\
    \sigma_{13}
    \end{pmatrix}= \frac{E}{(1+\nu)(1-2\nu)}\begin{pmatrix}
    (1-\nu) &  \nu & \nu & 0 & 0 & 0 \\
    \nu & (1-\nu) & \nu & 0 & 0 & 0 \\
    \nu & \nu & (1-\nu) & 0 & 0 & 0 \\
     &  &  & \frac{1-2\nu }{2} & 0 & 0 \\
     &  \text{SYM}&  &  & \frac{1-2\nu }{2} & 0 \\
     &  &  &  &  & \frac{1-2\nu }{2}
    \end{pmatrix}
    \begin{pmatrix}
    \varepsilon_{11} \\
    \varepsilon_{22} \\
    \varepsilon_{33} \\
    \gamma_{12} \\
    \gamma_{23} \\
    \gamma_{13}
    \end{pmatrix}
\end{equation}

with $\gamma_{ij} = 2 \, \varepsilon_{ij}$, which we can thus rewrite as 

\begin{equation}
\sigma_{ij} = \frac{E}{(1+\nu)(1-2\nu)} \left[ (1 - 2\nu) \varepsilon_{ij} + \nu \, \varepsilon_{\ell \ell} \, \delta_{ij} \right]
\end{equation}

\begin{equation}
\varepsilon_{ij} = \frac{1}{E} \left[ (1+\nu) \sigma_{ij} - \nu \, \sigma_{\ell \ell} \, \delta_{ij} \right]
\end{equation}

\subsubsection{Equilibrium equation using displacements \underline{\textbf{u}}}

Navier Equation :
\begin{comment}
 \begin{equation}
    \frac{E}{2(1+\nu)} \left( \frac{\partial^2u_i}{ \partial x_{j} ^2} + \frac{1}{1-2\nu} \, \frac{\partial^2u_i}{\partial x_i \partial x_j} \right) + F_i = 0
\end{equation}   
\end{comment}

\begin{equation}
    \frac{E}{2(1+\nu)} \left( D_{jj} \, u_i + \frac{1}{1-2\nu} D_{ij} \, u_i \right) + F_i = 0
\end{equation}

\subsubsection{Boundary conditions}

\begin{equation}
    \begin{pmatrix}
    u \\
    v \\
    w
    \end{pmatrix} 
    = 
    \begin{pmatrix}
    u_0 \\
    v_0 \\
    w_0
    \end{pmatrix}
\end{equation}
on $\Gamma_u$, the clamped surface, where displacements are imposed.

\begin{equation}
    T_j = n_i \sigma_{ij}
\end{equation}
with $q$ and $\mu q$ on $\Gamma_q$, where forces are applied.

\begin{equation}
    \begin{pmatrix}
    u \\
    v \\
    w
    \end{pmatrix} 
    = 
    \begin{pmatrix}
    u_0 \\
    v_0 \\
    w_0
    \end{pmatrix}
    \quad \text{on } \Gamma_u, \text{ the clamped surface, where displacements are imposed.}
\end{equation}

\begin{equation}
    T_j = n_i \sigma_{ij} \quad \text{with } T_j \text{ = q or $\mu$q on } \Gamma_q, \text{ where forces are applied. }
\end{equation}


\subsection{Simplifying hypotheses for an analytical solution}
In addition to the general hypotheses in subsection \ref{sec: SoM general hyp}, in order to study this problem anatically to get a first approximation of the solution, we add the following hypotheses : 
\begin{itemize}[itemsep = 1pt]
    \item Plane stress state ( applies for the entire problem)
    \item No body forces ( e.g. gravity)
    \item Linear elastic domain ( applies for the entire problem, we are searching $q_{max}$ such that we remain in the elastic domain)
    \item The ellipse at the bottom of the piece is replaced by an equivalent circle ( see details below)
    \item Stress concentration factors at fillet ( \textbf{congé de raccordement => anglais ?}), direction changes,... are neglected 
    \item Conservative Loading
\end{itemize}




\section{Analysis with the Finite Element Method}


\subsection{Optimization}


\vfill

\section*{Students' contributions}
Mr. Goose and Mrs. Goose worked together to understand the problem and write the numerical codes. The summary of the problem and Q2 were written up by Mr. Goose, Q1 and Q3 were completed by Mrs. Goose. Both students corrected the final report.

\newpage
\bibliography{mybib}


\end{document}
%%% Local Variables: 
%%% mode: latex
%%% TeX-master: t
%%% End: 
